\documentclass[12pt]{report}
\usepackage[utf8]{inputenc}
\begin{document}
\title{Transmission d'electricité sans fil par le biais de systemes de tranfert d'energie inductive (IPTS)}
\author{Blaise Ribon, Léo Boudoin, Quentin Boyer}
\date{Décembre 2014}
\maketitle
\begin{abstract}
	Suite a l'experience menée en 2007 au MIT , nous savons qu'il est possible de transmettre de l'electricité à travers de moyennes distances, de l'ordre de 5m. Ce type de transmissions d'electricité pourrait simplifier les reseaux electriques domestiques étant donné le nombre de cables demandés par chaque appareil electronique, qui proliferent. Mais nous verrons que cette technologie et celles semblables se heurtent à des freins majeurs dans la pratique et que leur mise en place est assez complexe.
\end{abstract}
\section{Sommaire}
\chapter{Définition et Utilisation de l'electricité} %Titre a changer je pense%
\section{Historique de l'electricité}
\section{Avancées technologiques et Utilisation}
\chapter{Raisons de la transmission de l'electricité par des solutions non cablées} %Titre encore plus pété%
\section{--TODO--}
\chapter{Les Systemes de transfert d'energie inductive (IPTS), technologie sans fil} % Je crois que celui la va par contre%
\section{Presentation des technologies présentes}
\subsection{Solution majoritaire actuelle : Les technoloogies cablées}
\subsection{Une solution IPTS limitée : Le systeme de couplage magnetique par resonnace (CMRS)}
\subsection{Une solution IPTS assez fiable : Une transmission utilisant des \{Dipole coils\}}
\section{--TODO--} %Technolologie choise%
\section{Avantages et limitations de --TODO--}
\section{Technlogies alternatives pour transmetre de l'energie}
\chapter{References et sources principales}
\section{Articles scientifiques}
\section{--Et les autres trucs--}
\end{document}
