\documentclass[12pt]{report}
\usepackage[utf8]{inputenc}
\usepackage[frenchb]{babel}
\begin{document}

\title{Transmission d'electricité sans fil par le biais de systemes de tranfert d'energie inductive (IPTS)}
\author{Blaise Ribon, Léo Boudoin, Quentin Boyer}
\date{Décembre 2014}
\maketitle

\begin{abstract}
	Suite a l'experience menée en 2007 au MIT , nous savons qu'il est possible de transmettre de l'electricité à travers de moyennes distances, de l'ordre de 5m. Ce type de transmissions d'electricité pourrait simplifier les reseaux electriques domestiques étant donné le nombre de cables demandés par chaque appareil electronique, qui proliferent. Mais nous verrons que cette technologie et celles semblables se heurtent à des freins majeurs dans la pratique et que leur mise en place est assez complexe.
\end{abstract}

\tableofcontents

\chapter{Définition et Utilisation de l'electricité} %Titre a changer je pense%
\section{Historique de l'electricité}
\section{Avancées technologiques et Utilisation}

\chapter{Raisons de la transmission de l'electricité par des solutions non cablées} %Titre encore plus pété , Mais en fait on pourrait faire la demarche de projet sur la transimission d'electicté tout court en preseantant tour a tour les solutions cablées et non cablées , à voir%
\section{--TODO--}

\chapter{Les technologies de transmissions d'electricité : cablées et sans-fil} %A voir si on insiste autant sur les cables%
\section{Presentation des technologies présentes}
\subsection{Solution majoritaire actuelle : Les technoloogies cablées}
	La solution de transmission d'electricité la plus utilisée au monde est sans contestation possible le cable electrique , ceci étant du à un faible coût (jusqu'a 1\$ le métre) , son très haut rendement puisque celui ci avoisine les 100\% sur les distances courtes avec de faibles puissances. En plus d'etre simple , elle n'est pas lourde en terme d'installation puisque les cables peuvent etre facilement mis dans les murs à la constuction d'un nouveau batiment, etre mis dans des gaines si l'on veut en rajouter ensuite et plus simplement on peut utiliser le systeme des prises pour les appareils temporaires et ponctuels. Grâce à ses avantages incontesables elle est devenue le standard , mais ceci entraîne un probleme non negligable qu'es la densité importante des cables electriques à proximité des appareils electroniques. %Ce texte est a ettofer avec des formules et autres explications plus détaillées sur les cables%
\subsection{Une solution IPTS limitée : Le systeme de couplage magnetique par resonnace (CMRS)}
\subsection{Une solution IPTS assez fiable : Une transmission utilisant des \{Dipole coils\}}
\section{--TODO--} %Technolologie choise%
\section{Avantages et limitations de --TODO--}
\section{Technlogies alternatives pour transmetre de l'energie}

\chapter{References et sources principales}
\section{Articles scientifiques}
\section{--Et les autres trucs--}
\end{document}
